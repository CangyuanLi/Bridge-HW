\documentclass[14pt]{extreport}
\usepackage{amsmath}
\usepackage{amssymb}
\usepackage{enumitem}
\usepackage[a4paper, total={7in, 10in}]{geometry}
\usepackage{graphicx}
\usepackage[utf8]{inputenc}
\usepackage{subfig}

\newcommand{\homework}[9]{
    \noindent
    \begin{center}
        \framebox{
            \vbox{
                \hbox to 6.50in { {\bf NYU Computer Science Bridge to Tandon Course} \hfill Spring 2022 }
                \vspace{4mm}
                \hbox to 6.50in { {\Large \hfill Homework 6  \hfill} }
                \vspace{3mm}
                \hbox to 6.50in { \underline{Name(s)} \hfill \hspace{17mm} \underline{NetID(s)} \hfill \hfill }
                \vspace{2mm}
                \hbox to 6.50in { {#2} \hfill \hspace{15mm}{#3} \hfill \hfill \hfill \hfill \hfill }
                \hbox to 6.50in { {#4} \hfill \hspace{24mm}{#5}  \hfill \hfill \hfill }
                \hbox to 6in { {#6} \hfill \hspace{35mm}{#7} \hfill \hfill \hfill }
                \hbox to 6.50in { {#8} \hfill \hspace{31mm}{#9}  \hfill \hfill \hfill}
            }
        }
    \end{center}
    \vspace*{4mm}
}

\newcommand{\ddfrac}[2]{\frac{\displaystyle #1}{\displaystyle #2}}
\newcommand{\eq}[0]{\llap{\( \Leftrightarrow \) \qquad}}
\newcommand{\answer}[0]{\medskip \textbf{Answer:} \medskip}
\newcommand{\union}[0]{\cup}
\newcommand{\intersect}[0]{\cap}
\newcommand{\sumn}[0]{\( \sum\limits_{i=1}^n \)}
\newcommand{\limn}[0]{\( \lim_{n \to \infty} \)}
\newcommand{\limt}[0]{\( \lim_{t \to \infty} \)}
\newcommand{\E}[0]{\mathbb{E}}
\newcommand{\R}[0]{\mathbb{R}}
\newcommand{\Z}[0]{\mathbb{Z}}

\begin{document}

\homework{1}{Rishie Nandhan Babu}{rb5291@nyu.edu}{Amulya Thota}{ast9920@nyu.edu}{Cangyuan Li}{cl4220@stern.nyu.edu}{Daniel Lim}{dl3009@nyu.edu}
\newpage

\section*{Question 7:}

\begin{enumerate}[label=(\alph*)]

\item Zybooks Exercise 6.1.5, sections b-d
A 5-card hand is dealt from a perfectly shuffled deck of playing cards. What is the probability of each of the following events?\\


b) What is the probability that the hand is a three of a kind? A three of a kind has 3 cards of the same rank. The other two cards do not have the same rank as each other and do not have the same rank as the three with the same rank.\\

Answer:\\
We should calculate the probability for the three of a kind and probability for selecting the set of other two cards.\\

Number of possibilities for three of a kind:
We first choose a rank from 13 ranks; and the choice for 3 cards would be from 4 suits of the rank.\\
So, outcomes $ = 13 * 4C3 $
\newline

Selecting other two cards:
We select 2 ranks from remaining 12 $\left( 12C2 \right)$
Each card will have 4 suits. So we multiply by 4 for first and again 4 for second card. \\
Outcomes $ = 12C2 * 4 * 4 $ \\

Overall, number of outcomes for three of a kind $ = 13 * 4C3 * 12C2 * 4 * 4 = 54912$\\

Probability $= 54912 / 52C5 $ \\
(52C5 is the total outcomes for 5 card hand from a standard deck of 52 cards. $52C5 = 2598960$)\\ 

Probability $ = 54912 / 2598960 = 0.02112845138$
\newline

\newpage

c) What is the probability that all 5 cards have the same suit?\\

Answer:\\
There are 13 ranks and 4 suits to choose from. \\
Whatever suit we select, we choose 5 ranks out of 13 to pick the 5 cards.\\
There will be 4 options for first card, but since all cards have to be of the same suit, there is only 1 option for each remaining card.\\

Number of outcomes $ = 13C5 * 4 = 1287$ \\

Probability $ = 1287 / total outcomes = 1287 / 52C5 = 1287 / 2598960 \\
= 0.00049519807$
\newline

d) What is the probability that the hand is a two of a kind? A two of a kind has two cards of the same rank (called the pair). Among the remaining three cards, not in the pair, no two have the same rank and none of them have the same rank as the pair. 
\newline

Answer:\\
We should calculate the probability for the two of a kind and probability for selecting the set of other three cards.\\

Number of possibilities for two of a kind:\\
We first choose a rank from 13 ranks; and the choice for 2 cards would be from 4 suits of the rank.\\
So, outcomes $ = 13 * 4C2 $
\newline

Selecting other three cards:\\
We select 3 ranks from remaining 12 $\left( 12C3 \right)$
Each card will have 4 suits. So we multiply by 4 for each card. \\
Outcomes $ = 12C3 * 4 * 4 * 4 $ \\

Overall, number of outcomes for three of a kind $ = 13 * 4C2 * 12C3 * 4 * 4 * 4 = 366080$\\

Probability $= 366080 / 52C5 = 366080 / 2598960 $ \\
 $= 0.14085634253$
 \newpage
 
\item Zybooks Exercise 6.2.4, sections a-d: \\
A 5-card hand is dealt from a perfectly shuffled deck of playing cards. What is the probability of each of the following events?\\

a) The hand has at least one club.\\

Answer:\\
Outcomes for cards with atleast one club is equal to total number of five card hands minus those with zero clubs. \\

Total number of five hand cards $ 52C5 = 2598960$\\
Those with zero clubs: we are now choosing from 13 ranks and 3 suits in each rank; so total 39 cards. \\
Number of possible five card hands with zero clubs $ = 39C5 $ \\
Number of five card hands with atleast one club $ = 52C5 - 39C5$ \\

Probability $= \left(52C5 - 39C5\right) / 52C5$ \\
    $ = 1 - \left(39C5 / 52C5\right) = 1 - \left( 575757/ 2598960\right)$ \\
    $ = 1 - 0.22153361344 = 0.77846638656$
    \newline
    
b) The hand has at least two cards with the same rank.\\

Answer:\\
Outcomes for cards with atleast atleast two cards of same rank is equal to total number of five card hands minus those with no cards of same rank. \\

Total number of five hand cards $ 52C5 = 2598960$\\
Those with no cards of same rank : we are now choosing 5 distinct cards(ranks) from 13 ranks and each card has option for 4 suits. So $ 13C5 * 4^{5} $ outcomes \\
Number of possible five card hands with atleast 2 cards of the same rank $ = 52C5 - (13C5 * 4^{5}) = 2598960 - 1317888 = 1281072 $\\

Probability $ = 1281072 / 52C5 = 1281072 / 2598960 $ \\
 $ = 0.49291716686$
\newline
\newpage

c) The hand has exactly one club or exactly one spade\\

Answer:\\

Outcomes for exactly one club: 13 rank options for one club. The remaining 4 cards will be chosen from the remainder of cards (reminder $= 13 ranks * 3 suits = 39 cards$) \\
So, outcomes $ = 13 * 39C4 $ \\
Outcomes for exactly one spade will also be calculated similarly = $ = 13 * 39C4 $ \\

Outcomes for exactly one club or exactly one spade: we should add above two numbers $ = (13 * 39C4) + (13 * 39C4) = 2 * 13 * 39C4 = 2138526 $\\

However, in calculating above, we counted a set of cards twice. This set is the cards that have exactly one club and exactly one spade. So, we calculate this and subtract from above sum. \\

Exactly one club and exactly one spade:\\
13 rank options for one club, 13 rank options for one spade. The remaining 3 cards do not have clubs or spades: choices $ = 13 ranks * 2 suits = 26 $. Choosing 3 cards from this is 26C3. \\
So outcomes for exactly one club and exactly one spade $ = 13 * 13 * 26C3  = 169 * 2600 = 439400 $ \\

Subtracting this from above sum:\\
Outcomes for exactly one club or exactly one spade $ = 2138526 - 439400 $ \\
 $ = 1699126 $ \\

Probability $ = 1699126 / 52C5 = 1699126 / 2598960 $ \\
 $= 0.6537715086 $ \\

\newpage
d) The hand has at least one club or at least one spade.\\

Answer: \\
Number of five card hands with atleast one club or atleast one spade is equal to total number of five card hands minus those with no clubs and no spades. \\

To calculate the outcomes for cards with no clubs and no spades: we have 2 suit options and 13 rank options for each suit. \\
So the available pool = 13 * 2 = 26 cards.\\
To choose five card hands with no clubs and no spades, outcomes  $= 26C5$ \\

Number of five card hands with atleast one club or atleast one spade $ = 52C5 - 26C5 $ \\

Probability $ = \left(52C5 - 26C5\right) / 52C5 $ \\
 $= 1 - \left(26C5 / 52C5\right) = 1 - \left(65780 / 2598960 \right) =  1- 0.02531012404$ \\
  $ = 0.97468987596 $ \\
  
  

\end{enumerate}
\newpage

\section*{Question 8:}

\begin{enumerate}[label=(\alph*)]
    
    \item[(a)] ZyBooks Exercise 6.3.2, sections a-e. The letters {a, b, c, d, e, f, g} are put in a random order. Each permutation is equally likely. Define the following events: 
    \begin{itemize}
        \item[A]: The letter b falls in the middle (with three before it and three after it)
        \item[B]: The letter c appears to the right of b, although c is not necessarily immediately to the right of b. For example, "agbdcef" would be an outcome in this event. 
        \item[C]: The letters "def" occur together in that order (e.g. "gdefbca")
    \end{itemize}
    
        \begin{enumerate}
            \item Calculate the probability of each individual event. That is, calculate p(A), p(B) and p(C). 
                
                \answer \\
                For A: \\
                The number of outcomes in the sample space, $|S| = 7!$\\
                The number of outcomes in event A, where the letter b is locked to the middle position would be the number of permutations that the remaining 6 letters have, $|B|=6!$\\
                $\therefore p(A) = \frac{|A|}{|S|} = \frac{6!}{7!} = \frac{1}{7}$\\
                
                For B:\\
                The letter b either comes before c or after it. The union of these two events are a universal set, so probability of either would total to 1. Hence, the probability that c appears to the right of b would be $\frac{1}{2}$. \\
                $\therefore p(B) = \frac{|B|}{|S|} = \frac{1}{2}$\\
                
                For C: \\
                Grouping the letters d, e and f as one element "def", we can run permutations on the effective number of remaining elements (5). \\
                $\therefore p(C) = \frac{|C|}{|S|} = \frac{5!}{7!} = \frac{1}{42}$\\
                
                \item What is $p(A|C)$?
                
                \answer \\
                \begin{align*}
                    p(A|C)  &=  \frac{|A \intersect C|}{|C|}\\
                            &   \text{With b in middle, def can only be before or after b.} \\
                            &   \text{With 'defbxxx' or 'xxxbdef', the x's can have 6 permutations}\\
                            &   \text{(3!), for each of these two events.}\\
                            &=  \frac{3!\cdot2}{5!}\\
                            &=  \frac{1}{10}\\
                \end{align*}
                
                \item What is $p(B|C)$?
                
                \answer \\
                \begin{align*}
                    p(B|C)  &=  \frac{|B \intersect C|}{|C|}\\
                            &   \text{Numerator will have def occurring together, so total permutations} \\
                            &   \text{of the set would be = 5!, for the 5 elements after grouping def}\\
                            &   \text{Half of these are events where c is right of b, and} \\
                            &   \text{other half, b is right of c.}\\
                            &=  |B \intersect C| = \frac{120}{2} = 60 \rightarrow (equation1)\\
                            &=  |C| = 5! = 120 \rightarrow (equation2)\\
                            &=  (equation1)/(equation2) = \frac{60}{120}\\
                            &=  \frac{1}{2}\\
                \end{align*}
    
                \item What is $p(A|B)$?
                
                \answer \\
                \begin{align*}
                    p(A|B)  &=  \frac{|A \intersect B|}{|B|}\\
                            &   \text{With b locked in the middle (4th place), then c has 4} \\
                            &   \text{possible positions, in 1st, 2nd, 3rd, 5th, 6th and 7th.}\\
                            &   \text{Two of these are left of b, two are right of b.} \\
                            &   \text{Hence, the number of outcomes would be half of 6!}\\
                            &=  |A \intersect B| = \frac{1}{2} \times 6! \rightarrow (equation 1)\\
                            &=  |B| = \frac{1}{2} \times 7! \rightarrow (equation 2)\\
                            &=  (equation 1) / (equation 2)\\
                            &=  \frac{6!}{7!} = \frac{1}{7}\\
                \end{align*}
                
                \item Which pairs of events among A, B, and C are independent? 
                
                \answer \\
                In order to verify that two events are independent, we must have $p(E|F)=p(E)$
                
                $p(A|C) = \frac{1}{10} \neq p(A) = \frac{1}{7} \therefore$ A and C are not independent.
                
                $p(B|C) = \frac{1}{2} \Leftrightarrow p(B) = \frac{1}{2} \therefore$ B and C are independent.
                
                $p(A|B) = \frac{1}{7} \Leftrightarrow  p(A) = \frac{1}{7} \therefore$ A and B are independent.\\
        \end{enumerate}
        
    \item[(b)] ZyBooks Exercise 6.3.6, sections b-c. A biased coin is flipped 10 times. In a single flip of the coin, the probability of heads is 1/3 and the probability of tails is 2/3. The outcomes of the coin flips are mutually independent. What is the probability of each event?
    
        \begin{enumerate}
            \item[(b)] The first 5 flips comes up heads. The last 5 flips come up tails.  
            
                \answer\\
                Probability of first 5 flips being heads = $(\frac{1}{3})^{5}$\\
                Probability of last 5 flips being tails = $(\frac{2}{3})^{5}$\\
                $\therefore p = (\frac{1}{3})^{5} \times (\frac{2}{3})^{5}$\\
                
            \item[(c)] The first flip comes up heads. The rest of the flips come up tails.
            
                \answer\\
                Probability of first flip being heads = $(\frac{1}{3})^{1}$\\
                Probability of last 9 flips being tails = $(\frac{2}{3})^{9}$\\
                $\therefore p = (\frac{1}{3})^{1} \times (\frac{2}{3})^{9}$\\
                
        \end{enumerate}
    
    \item[(c)] ZyBooks Exercise 6.4.2, section a
    
        \begin{enumerate}
            \item Assume that you have two dice, one of which is fair, and the other is biased toward landing on six, so that 0.25 of the time it lands on six, and 0.15 of the time it lands on each of 1, 2, 3, 4 and 5. You choose a die at random, and roll it six times, getting the values 4, 3, 6, 6, 5, 5. What is the probability that the die you chose is the fair die? The outcomes of the rolls are mutually independent.
            
                \answer\\
                    \text{Using Bayes' Theorem below:}\\
                    \begin{equation*}
                    p(F|X)=\frac{p(X|F)p(F)}{p(X|F)p(F)+p(X|\overline{F})p(\overline{F})}
                    \end{equation*}
                    Where $X$ is the observed event of \{4, 3, 6, 6, 5, 5\}\\
                    $F$ = fair die and $\overline{F}$ = biased die.\\
            
                    For fair die, $p(X_{1}) = p(X_{2}) = p(X_{3}) = p(X_{4}) = p(X_{5}) = p(X_{6}) = \frac{1}{6}$\\
                    For biased die, $p(X_{1}) = p(X_{2}) = p(X_{3}) = p(X_{4}) = p(X_{5}) = 0.15$,\\
                    $p(X_{6}) = 0.25$\\
                    
                    \begin{align*}
                        p(F)    &= p(\overline{F}) = \frac{1}{2}\\
                                \\
                        p(X|F)  &= \text{on a fair die, each number from 1-6 has a }\frac{1}{6}\\
                                &  \text{ chance, with 6 rolls in the event, so} \\
                                &= (\frac{1}{6})^{6}\\
                                \\
                        p(X|\overline{F}) &= (0.15)^{4} \times (0.25)^{2} \approx 0.00003164\\    
                    \end{align*}
                    
                    \begin{equation*}
                    p(F|X)=\frac{(\frac{1}{6})^{6} \times \frac{1}{2}}{((\frac{1}{6})^{6} \times \frac{1}{2}) + (0.00003164 \times \frac{1}{2})}
                    \approx 0.40385\\
                    \end{equation*}
                    
        \end{enumerate}

\end{enumerate}
\newpage    

\section*{Question 9:}

\subsubsection*{Section A: zyBooks Exercise 6.5.2; a-b}

A hand of 5 cards is dealt from a perfectly shuffled deck of playing cards. Let the random variable \( A \) denote the number of aces in the hand

\begin{enumerate}

    \item[(a)] What is the range of \( A \)?
    
        \answer

        The range is \( \{ 0, 1, 2, 3, 4 \} \). There are at most four Aces in a deck. It is also possible to draw 0 Aces.

    \item[(b)] Give the distribution over the random variable \( A \).
    
        \answer

        Step 1: The number of five-card hands in the deck is \( \binom{52}{5} \). The number of hands without an Ace (since there are four Aces in a deck) is then \( \binom{52 - 4}{5} = \binom{48}{5} \). So the probability of getting 0 aces in a five-card draw is 
        \[ \frac{\binom{48}{5}}{\binom{52}{5}} = \frac{12}{13} \]

        \bigskip

        Step 2: If you have a hand with only one Ace, you have four choices for your Ace. Then there can be no other Aces, so there are 48 cards left to choose from for the rest of the four cards. So the probability of getting 1 Ace is 
        \[ \frac{\binom{4}{1} \binom{48}{4}}{\binom{52}{5}} = \frac{60}{13} \]

        \bigskip

        Step 3: If you have a hand with two Aces, you have again 4 Aces to choose from and now pick two from those 4. Then there are three cards left for the rest. A similar pattern (hypergeometric) holds for the rest. So 
        \[
            \frac{\binom{4}{2} \binom{48}{3}}{\binom{52}{5}} = \frac{40}{13}
        \]

        \bigskip

        Step 4: Three Aces
        \[
            \frac{\binom{4}{3} \binom{48}{2}}{\binom{52}{5}} = \frac{40}{13}
        \]

        \bigskip

        Step 5: Four Aces
        \[
            \frac{\binom{4}{4} \binom{48}{1}}{\binom{52}{5}} = \frac{60}{13}
        \]

        So the distribution is
        \[
            \left\{ 
                \left( 0, \frac{12}{13} \right),
                \left( 1, \frac{60}{13} \right),
                \left( 2, \frac{40}{13} \right),
                \left( 3, \frac{40}{13} \right),
                \left( 4, \frac{60}{13} \right)
            \right\}
        \]

\end{enumerate}

\subsubsection*{Section B: zyBooks Exercise 6.6.1; a}

\begin{enumerate}

    \item[(a)] Two student council representatives are chosen at random from a group of 7 girls and 3 boys. Let \( G \) be the random variable denoting the number of girls chosen. What is \( \mathbb{E}[G] \)?
    
        \answer

        There can be 0, 1, or 2 girls chosen. Using the same logic as 6.5.2 (b), if there is only 1 girl on the council, draw 1 from the pool of 7 and draw 1 boy from the pool of 3, and so on (hypergeometric). So
        \begin{align*}
            \mathbb{E}[G] &= \sum_{r \in G(S)} r \cdot p(G = r) \\
                 &= 0 \cdot p(G = 0) + 1 \cdot \frac{\binom{7}{1} \binom{3}{1}}{\binom{10}{2}} + 2 \cdot \frac{\binom{7}{2} \binom{3}{0}}{\binom{10}{2}} \\
                 &= 0 + 1 \cdot \frac{21}{45} + 2 \cdot \frac{21}{45} \\
                 &= 1.4
        \end{align*}

\end{enumerate}

\subsubsection*{Section C: zyBooks Exercise 6.6.4; a-b}

\begin{enumerate}
    
    \item[(a)] A fair die is rolled once. Let \( X \) be the random variable that denotes the square of the number that shows up on the die. For example, if the die comes up 5, then \( X = 25 \). What is \( \mathbb{E}[X] \)?
    
        \answer

        The possible values of \( X \) are \( \left\{ 1, 4, 9, 16, 25, 36 \right\} \). The chance of rolling any number (and thus getting some value of \( x \)) is \( 1 / 6 \). So,
        \begin{align*}
            \mathbb{E}[X] &= 1(\frac{1}{6}) + 4(\frac{1}{6}) + 9(\frac{1}{6}) + 16(\frac{1}{6}) + 25(\frac{1}{6}) + 36(\frac{1}{6}) \\
                 &= \frac{91}{6} \\
                 &\approx 15.166667
        \end{align*}

    \item[(b)] A fair coin is tossed three times. Let \( Y \) be the random variable that denotes the square of the number of heads. What is \( \mathbb{E}[Y] \)?
    
        \answer

        It is possible to get 0 heads, 1 head, 2 heads, or 3 heads. Then the possible values of \( Y \) are \( \left\{ 0, 1, 4, 9 \right\} \). The the chance of getting \( Y = 4 \) is the same as the chance of getting 2 heads, and so on (a one-to-one mapping from \# of heads to \( Y \)). The probability of getting \( k \) successes in \( n \) trials follows a binomial distribution, but since that is in zyBooks section 6.8, here we list out the possible combinations.
        \[
            \left\{ HHH \right\},
            \left\{ HHT \right\},
            \left\{ HTT \right\},
            \left\{ HTH \right\},
            \left\{ THH \right\},
            \left\{ THT \right\},
            \left\{ TTH \right\},
            \left\{ TTT \right\}
        \]
        So then the probability is just the number of occurences divided by 8.
        \begin{align*}
            \E[Y] &= 0(p) + 1 \frac{3}{8} + 4 \frac{3}{8} + 9 \frac{1}{8} \\
                  &= 3
        \end{align*}

\end{enumerate}

\subsubsection*{Section D: zyBooks Exercise 6.7.4; a}

\begin{enumerate}
    
    \item[(a)] A class of 10 students hang up their coats when they arrive at school. Just before recess, the teacher hands one coat selected at random to each child. What is the expected number of children who get his or her own coat?
    
        \answer

        Step 1: By the linearity of expectation, the expected number of children that get their own coat is equal to the sum of the expected values of each particular child getting their own coat. Define
        \begin{align*}
            X_i = 
            \begin{cases}
                1 \text{ if child \( i \) got their own coat}, & p(X_i = 1) = 1 / 10 \\
                0 \text{ if child \( i \) did not get their own coat}, & p(X_i = 0) = 9 / 10 \\
            \end{cases}
        \end{align*}

        There are 10 coats handed out randomly, so the chance of getting his / her own coat is \( 1 / 10 \). Then \( \E[X_i] = 1 \cdot \frac{1}{10}  + 0 \cdot \frac{9}{10} = \frac{1}{10} \)

        \bigskip

        Step 2: Using the linearity of expectations
        \begin{align*}
            \E[X] &= \sum_{i=1}^{10} \E[X_{i}] \\
                  &= \frac{1}{10} + \frac{1}{10} + \cdots \\
                  &= 10 \cdot \frac{1}{10} \\
                  &= 1
        \end{align*}

\end{enumerate}
\newpage

\section*{Question 10:}
\begin{enumerate}[label=(\alph*)]

\item  Exercise	6.8.1, sections a-d
\newline

a) What is the probability that out of 100 circuit boards made exactly 2 have defects?\\\\
Let the chosen circuit board having a defect be defined as success.\\\\
The parameters of a Bernoulli trial are as follows:
\begin{align*}
    p &= 0.01\\
    n &= 100\\
    k &= 2
\end{align*}
Using binomial expansion, we get,
\begin{align*}
        b\left ( 2; 100, 0.01 \right )  &=  \binom{n}{k}\cdot p^{k}\cdot q^{n-k}\\
                &=  \binom{100}{2}\cdot (0.01)^{2}\cdot (0.99)^{98}\\
                &\approx 0.1849
\end{align*}

b) What is the probability that out of 100 circuit boards made at least 2 have defects?\\

Let $p$ denote the probability that out of 100 circuit boards made, at least 2 have defects.\\\\
Probability of at least 2 defects = 1 - (Probability of 0 defects + Probability of 1 defect )\\\\
Using binomial expansion, we get,
\begin{align*}
        p  &=  1 - b\left ( 0; 100, 0.01 \right ) - b\left ( 1; 100, 0.01 \right )\\
                &= 1 - \binom{100}{0}\cdot (0.01)^{0}\cdot (0.99)^{100} - \binom{100}{1}\cdot (0.01)^{1}\cdot (0.99)^{99}\\
                &= 1-(0.99)^{100} - (0.99)^{99}\\
                &\approx 0.2642
\end{align*}
\newpage

c) What is the expected number of circuit boards with defects out of the 100 made?\\\\
Let $X$ be the random variable that denotes the number of circuit boards with defects out of 100 manufactured.
\begin{align*}
    E[X]    &=  n.p\\
            &=  100 \times 0.01\\
            &=  1
\end{align*}

d) Now suppose that the circuit boards are made in batches of two. Either both circuit boards in a batch have a defect or they are both free of defects. The probability that a batch has a defect is 1\%. What is the probability that out of 100 circuit boards (50 batches) at least 2 have defects? What is the expected number of circuit boards with defects out of the 100 made? How do your answers compared to the situation in which each circuit board is made separately?\\

Since the circuit boards are now made in batches of 2 and that both circuit boards in a batch are defective/free of defects at the same time, the number of Bernoulli trials effectively gets reduced by a factor of 2.\\
i.e. $n = 50$\\

Let $p$ denote the probability that in 100 circuit boards (50 batches) made, at least 2 circuit boards (1 batch) has defects.
\begin{align*}
        p  &=  1 - b\left ( 0; 50, 0.01 \right )\\
                &= 1 - \binom{50}{0}\cdot (0.01)^{0}\cdot (0.99)^{50}\\
                &= 1-(0.99)^{50}\\
                &\approx 0.3949
\end{align*}

Let $X$ be a random variable that denotes the number of circuit boards with defects out of the 100 manufactured and $Y$ be the random variable denoting the number of batches with defects out of 50 made.
\begin{align*}
    \therefore E[X]    &= E[2Y]\\
            &= 2E[Y]\\
            &=  2.n.p\\
            &=  2 \times 50 \times 0.01\\
            &=  1
\end{align*}

Compared to parts (b), (c), the E[X] is the same, but the probability that at least 2 circuit boards are defective are different. This is because the circuit boards individually are independent but the circuit boards in batches are not independent (i.e. each circuit board being defective in a batch is dependent on the other).\\

\item  Exercise	6.8.3,	section b
\newline

b) What is the probability that you reach an incorrect conclusion if the coin is biased?\\

Following observations can be made from the question:
\begin{itemize}
  \item Incorrect conclusion of the coin being biased occurs when we conclude that the coin is fair
  \item If the number of heads is at least 4, we conclude that the coin is fair
  \item Probability that the number of heads is at least 4 is 1 - probability that the number of heads is less than or equal to 3
\end{itemize}

Let $p$ denote the probability that the number of heads is at least 4.
\begin{align*}
        p  &=  1 - b\left ( 0; 10, 0.3 \right ) - b\left ( 1; 10, 0.3 \right ) - b\left ( 2; 10, 0.3 \right ) - b\left ( 3; 10, 0.3 \right )\\
                &= 1 - \binom{10}{0}\cdot (0.3)^{0}\cdot (0.7)^{10} - \binom{10}{1}\cdot (0.3)^{1}\cdot (0.7)^{9} \\
                &- \binom{10}{2}\cdot (0.3)^{2}\cdot (0.7)^{8} - \binom{10}{3}\cdot (0.3)^{3}\cdot (0.7)^{7}\\
                &\approx 0.35
\end{align*}

\end{enumerate}
\newpage

\end{document}