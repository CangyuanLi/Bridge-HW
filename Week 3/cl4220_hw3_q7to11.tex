\documentclass[14pt]{extreport}
\usepackage{amsmath}
\usepackage{amssymb}
\usepackage[a4paper, total={7in, 10in}]{geometry}
\usepackage{graphicx}
\usepackage[utf8]{inputenc}

\title{Homework 3}
\author{Cangyuan Li}
\date{\today}

\newcommand{\ddfrac}[2]{\frac{\displaystyle #1}{\displaystyle #2}}
\newcommand{\eq}[0]{\llap{\( \Leftrightarrow \) \qquad}}
\newcommand{\answer}[0]{\medskip \textbf{Answer:} \medskip}
\newcommand{\union}[0]{\cup}
\newcommand{\intersect}[0]{\cap}
\newcommand{\sumn}[0]{\( \sum\limits_{i=1}^n \)}
\newcommand{\limn}[0]{\( \lim_{n \to \infty} \)}
\newcommand{\limt}[0]{\( \lim_{t \to \infty} \)}

\begin{document}

\maketitle

\subsection*{Question 7:}

\begin{enumerate}
    
    \item zyBooks Exercise 3.1.1; a-g
    
    \begin{itemize}
        \item \( A = \left\{ x \in \mathbb{Z}: \text{x is an integer multiple of 3} \right\} \)
        \item \( B = \left\{ x \in \mathbb{Z}: \text{x is a perfect square} \right\} \)
        \item \( C = \left\{ 4, 5, 9, 10 \right\} \)
        \item \( D = \left\{ 2, 4, 11, 14 \right\} \)
        \item \( E = \left\{ 3, 6, 9 \right\} \)
        \item \( F = \left\{ 4, 6, 16 \right\} \)
    \end{itemize}
    
        \begin{enumerate}
            \item[(a)] \( 27 \in A \)
            
                \answer

                True, since 27 is a multiple of 3

            \item[(b)] \( 27 \in B \)
            
                \answer

                False, since 27 is not a perfect square
            
            \item[(c)] \( 100 \in B \)
            
                \answer

                True, since 100 is a perfect square

            \item[(d)] \( E \subseteq C \lor C \subseteq E \)
            
                \answer

                False, since the statement translates to ``Every element of E is also an element of C or every element of C is an element of E''. The first part is False because \( 3 \notin C \), and the second part is False because \( 4, 5, 10 \notin E \).

            \item[(e)] \( E \subseteq A \)
            
                \answer

                True, since all of \( 3, 6, 9 \) are integer multiples of 3.

            \item[(f)] \( A \subset E \)
            
                \answer

                False. A proper subset translates to \( (A \subseteq E) \land (A \neq B) \). Since the cardinality of \( A \) is greater than the cardinality of \( E \), it is impossible for every element of \( A \) to be in \( E \), so the first condition of being a proper subset \( A \subseteq E \) is not satisfied.

            \item[(g)] \( E \in A \)
            
                \answer

                False. \( E \) is a subset of \( A \), but \( A \) does not contain the set \( E \).

        \end{enumerate}

    \item zyBooks Exercise 3.1.2; a-e
    
    \begin{itemize}
        \item \( A = \left\{ x \in \mathbb{Z}: \text{x is an integer multiple of 3} \right\} \)
        \item \( B = \left\{ x \in \mathbb{Z}: \text{x is a perfect square} \right\} \)
        \item \( C = \left\{ 4, 5, 9, 10 \right\} \)
        \item \( D = \left\{ 2, 4, 11, 14 \right\} \)
        \item \( E = \left\{ 3, 6, 9 \right\} \)
        \item \( F = \left\{ 4, 6, 16 \right\} \)
    \end{itemize}
    
        \begin{enumerate}
            
            \item[(a)] \( 15 \subset A \)
            
                \answer

                False. \( 15 \in A \), but it is an element so it cannot be a subset of another set.

            \item[(b)] \( \{15\} \subset A \)
            
                \answer

                True, since every element of the set (15) is in \( A \), and \( {15} \neq A \).

            \item[(c)] \( \emptyset \subset A \)
            
                \answer

                True, since the empty set is a subset of every set.

            \item[(d)] \( A \subseteq A \)
            
                \answer

                True, since every element of \( A \) is in \( A \).

            \item[(e)] \( \emptyset \in B \)
            
                \answer

                False, no element of \( B \) is a set.

        \end{enumerate}

    \item zyBooks Exercise 3.1.5; b, d
    
    Express each set using set builder notation. Then if the set is finite, give its cardinality. Otherwise, indicate that the set is infinite.
    
        \begin{enumerate}

            \item[(b)] \( \{ 3, 6, 9, 12, \dots \} \)
            
                \answer

                This is the set of all positive integer multiples of 3. 
                
                So \( A = \left\{ x \in \mathbb{Z}^{+}: x \mod 3 = 0 \right\} \). This set is infinite because \( \mathbb{Z}^+ \) is infinite.
            
            \item[(d)] \( \{ 0, 10, 20, 30, \dots 1000 \} \)
            
                \answer

                This is the set of non-negative integer multiples of 10, up to 1000.

                So \( A = \left\{ x \in \mathbb{Z}, (0 \leq x \leq 1000) \land (x \mod 10 = 0) \right\} \), and \( |A| = 101 \).

        \end{enumerate}

    \item zyBooks Exercise 3.2.1; a-k
    
    Let \( X = \{ 1, \{1\}, \{1, 2\}, 2, \{3\}, 4 \} \).
    
        \begin{enumerate}
            
            \item[(a)] \( 2 \in X \)
            
                \answer

                True, since 2 is an element of \( X \).

            \item[(b)] \( \{2\} \subseteq X \)
            
                \answer

                True, since every element of \{2\} is in \( X \).

            \item[(c)] \( \{2\} \in X \)
            
                \answer

                False, since \( X \) does not contain the set \{2\}.

            \item[(d)] \( 3 \in X \)
            
                \answer

                False, since 3 is not an element of \( X \).

            \item[(e)] \( \{1, 2\} \in X \)
            
                \answer

                True, since \{1, 2\} is a set in \( X \).

            \item[(f)] \( \{1, 2\} \subseteq X \)
            
                \answer

                True, since every element of \{1, 2\} is in \( X \).

            \item[(g)] \( \{2, 4\} \subseteq X \)
            
                \answer

                True, since every element of \{2, 4\} is in \( X \).

            \item[(h)] \( \{2, 4\} \in X \)
            
                \answer

                False, since \( X \) does not contain the set \{2, 4\}.

            \item[(i)] \( \{2, 3\} \subseteq X \)
            
                \answer

                False, since 3 is not an element of \( X \).

            \item[(j)] \( \{2, 3\} \in X \)
            
                \answer

                False, since \( X \) does not contain the set \{2, 3\}

            \item[(k)] \( |X| = 7 \)
            
                \answer

                False, \( X \) has 6 elements.

        \end{enumerate}

\end{enumerate}
\newpage

\subsection*{Question 8:}

zyBooks Exercise 3.2.4; b

\begin{enumerate}
    \item [(b)] Let \( A = \{1, 2, 3\} \). What is \( \left\{ X \in \mathcal{P}(A): 2 \in X \right\} \)?
    
        \answer

        Step 1: \( \mathcal{P}(A) = \left\{ \emptyset, \{1\}, \{2\}, \{3\}, \{1, 2\}, \{1, 3\}, \{2, 3\}, \{1, 2, 3\} \right\} \)

        \medskip

        Step 2: We want the sets in the powerset of \( A \) that contain the element 2. So \( \left\{ X \in \mathcal{P}(A): 2 \in X \right\} = \left\{ \{2\}, \{1, 2\}, \{2, 3\}, \{1, 2, 3\} \right\} \)
\end{enumerate}
\newpage

\subsection*{Question 9:}

\begin{enumerate}
    
    \item zyBooks Exercise 3.3.1; c-e
    
    Define the sets \( A, B, C, D \) as follows:
    \begin{itemize}
        \item \( A = \{-3, 0, 1, 4, 17\} \)
        \item \( B = \{-12, -5, 1, 4, 6\} \)
        \item \( C = \{x \in Z: x \text{ is odd}\} \)
        \item \( D = \{x \in Z: x \text{ is positive}\} \)
    \end{itemize}

        \begin{enumerate}
            
            \item[(c)] \( A \intersect C \)
            
                \answer

                All elements of \( A \) that are odd integers. So \( \{-3, 1, 17\} \)

            \item[(d)] \( A \union (B \intersect C) \)
            
                \answer

                Step 1: \( B \intersect C \) is all elements of \( B \) that are odd integers. So \( \{ -5, 1 \} \)

                \medskip

                Step 2: \( A \union (B \intersect C) \) is then \( \{ -3, 0, 1, 4, 17, -5 \} \). 

            \item[(e)] \( A \intersect B \intersect C \)  
            
                \answer

                From the previous question \( B \intersect C \) is \( \{ -5, 1 \} \). By the associative property we can do this first. Then \( A \intersect B \intersect C = \{ 1 \} \).

        \end{enumerate}

    \item zyBooks Exercise 3.3.3; a, b, e, f
    
    Use the following definitions to express each union or intersection given. You can use roster or set builder notation in your responses, but no set operations. For each definition, \( i \in \mathbb{Z}^{+} \)

    \begin{itemize}
        \item \( A_i = \left\{ i^0, i^1, i^2 \right\} \)
        \item \( B_i = \left\{ x \in \mathbb{R}: -i \leq x \leq \frac{1}{i} \right\} \)
        \item \( C_i = \left\{ x \in \mathbb{R}: \frac{-1}{i} \leq x \leq \frac{1}{i} \right\} \)
    \end{itemize}
    
        \begin{enumerate}
            
            \item[(a)] \( \bigcap_{i=2}^{5} A_i \)
            
                \answer

                \begin{itemize}
                    \item \( A_2 = \{1, 4, 8\} \)
                    \item \( A_3 = \{1, 9, 27\} \)
                    \item \( A_4 = \{1, 16, 64\} \)
                    \item \( A_5 = \{1, 25, 125\} \)
                \end{itemize}

                So \( \bigcap_{i=2}^{5} A_i = \{1\} \)
            
            \item[(b)] \( \bigcup_{i=2}^5 A_i \)
            
                \answer

                This is the union of \( A_2 \) to \( A_5 \). So \( \bigcup_{i=2}^5 A_i = \{ 1, 4, 8, 9, 27, 16, 64, 25, 125 \} \)

            \item[(e)] \( \bigcap_{i=1}^{100} C_i \)
            
                \answer

                For for the first three choices if \( i \), the set is all real numbers from -1 to 1, from -1/2 to 1/2, and from -1/3 to 1/3 respectively. The domain shrinks as \( i \) increases. Since this is the intersection of all the sets, we only look at the smallest possible domain. This is from \( -1 \) to \( 1/100 \). So \( \bigcap_{i=1}^{100} C_i = \left\{ x \in \mathbb{R}: \frac{-1}{100} \leq x \leq \frac{1}{100} \right\} \)

            \item[(f)] \( \bigcup_{i=1}^{100} C_i \)
            
                \answer

                Use the same logic as in (e), except we look at the largest possible domain. So \( \bigcup_{i=1}^{100} C_i = \left\{ x \in \mathbb{R}: -1 \leq x \leq 1 \right\} \).

        \end{enumerate}

    \item zyBooks Exercise 3.3.4; b, d
    
    Use the set definitions \( A = \{a, b\} \) and \( B = \{b, c\} \) to express each set below. Use roster notation in your solutions.
    
        \begin{enumerate}
            
            \item[(b)] \( \mathcal{P}(A \union B) \)
            
                \answer

                Step 1: \( A \union B = \{a, b, c\} \)

                \medskip

                Step 2: The powerset is then 
                
                \( \mathcal{P}(A \union B) = \left\{ \emptyset, \{a\}, \{b\}, \{c\}, \{a, b\}, \{a, c\}, \{b, c\}, \{a, b, c\} \right\} \)

            \item[(d)] \( \mathcal{P}(A) \union \mathcal{P}(B) \)
            
                \answer

                Step 1: 
                
                \( \mathcal{P}(A) = \left\{ \emptyset, \{a\}, \{b\}, \{a, b\} \right\} \)

                \( \mathcal{P}(B) = \left\{ \emptyset, \{b\}, \{c\}, \{b, c\} \right\} \)

                \medskip

                Step 2:

                \( \mathcal{P}(A) \union \mathcal{P}(B) = \left\{ \emptyset, \{a\},\{b\}, \{c\}, \{a, b\}, \{b, c\} \right\} \)

        \end{enumerate}
        
\end{enumerate}
\newpage

\subsection*{Question 10:}

\begin{enumerate}
    
    \item zyBooks Exercise 3.5.1; b-c

    Use the definitions for A, B, and C to answer the questions. Express the elements using n-tuple notation, not string notation. The sets A, B, and C are defined as follows:
    \begin{itemize}
        \item \( A \) = \{tall, grande, venti\}
        \item \( B \) = \{foam, no-foam\}
        \item \( C \) = \{non-fat, whole\}
    \end{itemize}
    
        \begin{enumerate}

            \item[(b)] Write an element from the set \( B \times A \times C \).
            
                \answer

                (foam, tall, non-fat)

            \item[(c)] Write the set \( B \times C \) using roster notation.
            
                \answer

                \( \left\{ (b, c): b \in B \land c \in C \right\} \)

                \{(foam, non-fat), (foam, whole), (no-foam, non-fat), (no-foam, whole)\}
        \end{enumerate}

    \item zyBooks Exercise 3.5.3; b, c, e
    
        \begin{enumerate}
            
            \item[(b)] \( \mathbb{Z}^2 \subseteq \mathbb{R}^2 \)
            
                \answer

                True, since \( Z \subseteq R \).

            \item[(c)] \( \mathbb{Z}^2 \intersect \mathbb{Z}^3 = \emptyset \)
            
                \answer

                True, since an \( n \)-tuple of size 2 does not equal an \( n \)-tuple of size 3.

            \item[(e)] For any three sets \( A, B, C \), if \( A \subseteq B \), then \( A \times C \subseteq B \times C \)
            
                \answer

                True, since if \( A \subseteq B \) then every element of \( A \) is in \( B \). 

        \end{enumerate}

    \item zyBooks Exercise 3.5.6; d-e
    
    Express the following sets using the roster method. Express the elements as strings, not n-tuples.
    
        \begin{enumerate}

            \item[(d)] \( \left\{ xy: x \in \{0\} \union \{0\}^2 \land y \in \{1\} \union \{1\}^2 \right\} \)
            
                \answer

                Step 1: \( \{0\}^2 = \{00\} \) and \( {1}^2 = \{11\} \). So \( x \) is in \( \{0, 00\} \) and \( y \) is in \( \{1, 11\} \)

                \medskip

                Step 2: \( \left\{ 01, 011, 001, 0011 \right\} \)

            \item[(e)] \( \left\{ xy: x \in \{aa, ab\} \land y \in \{a\} \union \{a\}^2 \right\} \)
            
                \answer

                \( \left\{ aaa, aaaa, aba, abaa \right\} \)

        \end{enumerate}

    \item zyBooks Exercise 3.5.7; c, f, g
    
    Use the following set definitions to specify each set in roster notation. Except where noted, express elements of Cartesian products as strings.
    \begin{itemize}
        \item \( A = \{a\} \)
        \item \( B = \{b, c\} \)
        \item \( C = \{a, b, d\} \)
    \end{itemize}

        \begin{enumerate}
            
            \item[(c)] \( (A \times B) \union (A \times C) \)
            
                \answer
                \begin{align*}
                    (A \times B) \union (A \times C) &= \{ab, ac\} \union \{aa, ab, ad\} \\
                                                    &= \{ab, ac, aa, ab, ad\} \\
                                                    &= \{ab, ac, aa, ad\}
                \end{align*}

            \item[(f)] \( \mathcal{P}(A \times B) \)
            
                \answer
                \begin{align*}
                    \mathcal{P}(A \times B) &= \mathcal{P}(\{ab, ac\}) \\
                        &= \{ \emptyset, \{ab\}, \{ac\}, \{ab, ac\} \}
                \end{align*}

            \item[(g)] \( \mathcal{P}(A) \times \mathcal{P}(B) \)
            
                \answer
                \begin{align*}
                    \mathcal{P}(A) \times \mathcal{P}(B) &= \left\{ \emptyset, \{a\} \} \times \{ \emptyset, \{b\}, \{c\}, \{b, c\} \right\} \\
                        &= \{ (\emptyset, \emptyset), (\emptyset, \{b\}), (\emptyset, \{c\}), (\emptyset, \{b, c\}), \\
                        & \qquad \qquad (\{a\}, \emptyset), (\{a\}, \{b\}), (\{a\}, \{c\}), (\{a\}, \{b, c\}) \}
                \end{align*}

        \end{enumerate}

\end{enumerate}
\newpage

\subsection*{Question 11:}

\begin{enumerate}
    
    \item zyBooks Exercise 3.6.2; b-c

    Use the set identities given in the table to prove the following new identities. Label each step in your proof with the set identity used to establish that step.

    
        \begin{enumerate}
            
            \item[(b)] \( (B \union A) \intersect (\overline{B} \union A) \)
            
                \answer
                \begin{align*}
                    (B \union A) \intersect (\overline{B} \union A) &= (B \intersect \overline{B}) \union A, \text{ by Distributive Law} \\
                        &= \emptyset \union A, \text{ by Complement Law} \\
                        &= A, \text{ by Identity Law}
                \end{align*}

            \item[(c)] \( \overline{A \intersect \overline{B}} = \overline{A} \union B \)
                
                \answer
                \begin{align*}
                    \overline{A \intersect \overline{B}} &= \overline{A} \union \overline{\overline{B}}, \text{ by De Morgan's Law} \\
                        &= \overline{A} \union B, \text{ by Double Complement}
                \end{align*}
            
        \end{enumerate}

    \item zyBooks Exercise 3.6.3; b, d
    
    A set equation is not an identity if there are examples for the variables denoting the sets that cause the equation to be false. Show that each set equation given below is not a set identity.
    
        \begin{enumerate}
            
            \item[(b)] \( A - (B \intersect A) = A \)
            
                \answer

                Let \( A \) be any set except the empty set, and let \( B = A \). Then \( B \intersect A = A \) by idempotency. Then \( A - A \) is the empty set. As a concrete example, \( A = \{1\}, B = \{1\} \) is a counterexample.

            \item[(d)] \( (B - A) \union A = A \)
            
                \answer

                Let \( B = \{1, 2\} \) and \( A = \{1\} \). Then:
                \begin{align*}
                    (B - A) \union A &= (\{1, 2\} - \{1\}) \union \{1\} \\
                                     &= \{2\} \union \{1\} \\
                                     &= \{2, 1\}
                \end{align*}

                Which does not equal \( A \).
                
        \end{enumerate}

    \item zyBooks Exercise 3.6.4; b-c
    
    The set subtraction law states that \( A - B = A \intersect \overline{B} \). Use the set subtraction law as well as the other set identities given in the table to prove each of the following new identities. Label each step in your proof with the set identity used to establish that step.
    
        \begin{enumerate}
            
            \item[(b)] \( A \intersect (B - A) = \emptyset \)
            
                \answer
                \begin{align*}
                    A \intersect (B - A) &= A \intersect (B \intersect \overline{A}), \text{ by Set Subtraction Law} \\
                        &= (B \intersect \overline{A}) \intersect A, \text{ by Commutative Law} \\
                        &= B \intersect (\overline{A} \intersect A), \text{ by Associative Law} \\
                        &= B \intersect \emptyset, \text{ by Complement Law} \\
                        &= \emptyset, \text{ by Domination Law}
                \end{align*}

            \item[(c)] \( A \union (B - A) = A \union B \)
            
                \answer
                \begin{align*}
                    A \union (B - A) &= A \union (B \intersect \overline{A}), \text{ by Set Subtraction Law} \\
                                     &= (A \union B) \intersect (A \union \overline{A}), \text{ by Distributive Law} \\
                                     &= (A \union B) \intersect \mathcal{U}, \text{ by Complement Law} \\
                                     &= A \union B, \text{ by Identity Law}
                \end{align*}
        \end{enumerate}
\end{enumerate}

\end{document}