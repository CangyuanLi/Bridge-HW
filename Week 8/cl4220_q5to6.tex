\documentclass[14pt]{extreport}
\usepackage{amsmath}
\usepackage{amssymb}
\usepackage{enumitem}
\usepackage[a4paper, total={7in, 10in}]{geometry}
\usepackage{graphicx}
\usepackage[utf8]{inputenc}
\usepackage{subfig}

\newcommand{\homework}[9]{
    \noindent
    \begin{center}
        \framebox{
            \vbox{
                \hbox to 6.50in { {\bf NYU Computer Science Bridge to Tandon Course} \hfill Spring 2022 }
                \vspace{4mm}
                \hbox to 6.50in { {\Large \hfill Homework 8 \hfill} }
                \vspace{3mm}
                \hbox to 6.50in { \underline{Name(s)} \hfill \hspace{17mm} \underline{NetID(s)} \hfill \hfill }
                \vspace{2mm}
                \hbox to 6.50in { {#2} \hfill \hspace{15mm}{#3} \hfill \hfill \hfill \hfill \hfill }
                \hbox to 6.50in { {#4} \hfill \hspace{24mm}{#5}  \hfill \hfill \hfill }
                \hbox to 6in { {#6} \hfill \hspace{35mm}{#7} \hfill \hfill \hfill }
                \hbox to 6.50in { {#8} \hfill \hspace{31mm}{#9}  \hfill \hfill \hfill}
            }
        }
    \end{center}
    \vspace*{4mm}
}

\newcommand{\ddfrac}[2]{\frac{\displaystyle #1}{\displaystyle #2}}
\newcommand{\eq}[0]{\llap{\( \Leftrightarrow \) \qquad}}
\newcommand{\answer}[0]{\medskip \textbf{Answer:} \medskip}
\newcommand{\union}[0]{\cup}
\newcommand{\intersect}[0]{\cap}
\newcommand{\sumn}[0]{\( \sum\limits_{i=1}^n \)}
\newcommand{\limn}[0]{\( \lim_{n \to \infty} \)}
\newcommand{\limt}[0]{\( \lim_{t \to \infty} \)}
\newcommand{\R}[0]{\mathbb{R}}
\newcommand{\Z}[0]{\mathbb{Z}}

\begin{document}

\homework{1}{Rishie Nandhan Babu}{rb5291@nyu.edu}{Amulya Thota}{ast9920@nyu.edu}{Cangyuan Li}{cl4220@stern.nyu.edu}{Daniel Lim}{dl3009@nyu.edu}
\newpage

\section*{Question 5:}

\subsection*{Section A:}

Use mathematical induction to prove that for any positive integer \( n \), 3 divides \( n^3 + 2n \). We use the formula \( (a + b)^3 = a^3 + 3a^2b + 3ab^2 + b^3 \)

\bigskip

\answer

Step 1: The base case is \( n = 1 \). \( 1^3 + 2(1) = 3 \), which is divisible by 3. We can find an integer \( m \) such that \( 3m = 3 \) (\( m = 1 \)).

\medskip

Step 2: For all \( k \geq 1 \), we want to show that 3 divides \( (k + 1)^3 + 2(k + 1) \).

\begin{align*}
    (k + 1)^3 + 2(k + 1) &= k^3 + 3k^2(1) + 3k(1^2) + 1^3 + 2k + 2 \\
                         &= k^3 + 3k^2 + 3k + 2k + 1 + 2 \\
                         &= k^3 + 2k + 3k^2 + 3k + 3 \\
                         &= 3z + 3k^2 + 3k + 3, \text{ by the inductive hypothesis}\\
                         &= 3(z + k^2 + k + 1)
\end{align*}

Since \( k, z \) are integers, the expression \( z + k^2 + k + 1 \) is an integer. So we can find an integer \( m = z + k^2 + k + 1 \) such that \( (k + 1)^3 + 2(k + 1) = 3m \).

\subsection*{Section B:}

Use strong induction to prove that any positive integer \( n \geq 2 \) can be written as a product of primes.

\answer

Step 1: The base case is \( k = 2 \) and \( k = 3 \). For \( k = 2, 2 \times 1 = 2 \). For \( k = 3, 3 \times 1 = 3 \). So the base case is satisfied.

\medskip

Step 2: For all \( k \geq 2 \), we want to show that \( k + 1 \) can be written as a product of two primes. If the number is prime, then it can be expressed as that number times 1. If that number is not prime, then

\begin{align*}
    k + 1 &= ab
\end{align*}

where \( a, b \) are integers. By the inductive hypothesis \( a, b \) can be expressed as a product of primes or are primes themselves. So \( k + 1 \) can be expressed as a product of primes.

\section*{Question 6:}

\subsection*{Section A: zyBooks Exercise 7.4.1; a-g}

Define \( P(n) \) to be the assertion that 
\[
    \sum_{j=1}^n j^2 = \frac{n(n + 1)(2n + 1)}{6}
\]

\begin{enumerate}
    
    \item[(a)] Verify that \( P(3) \) is true.
    
        \answer

        \begin{align*}
            \sum_{j=1}^3 j^2 &= 1^2 + 2^2 + 3^3 \\
                         &= 1 + 4 + 9 \\
                         &= 14
        \end{align*}

        \begin{align*}
            \frac{n(n + 1)(2n + 1)}{6} &= \frac{3(3 + 1)(2 \cdot 3 + 1)}{6} \\
                                       &= \frac{3(4)(7)}{6} \\
                                       &= 14
        \end{align*}
        
    \item[(b)] Express \( P(k) \).
    
        \answer

        \begin{align*}
            P(k) &= \sum_{j=1}^k j^2 \\
                 &= \frac{k(k + 1)(2k + 1)}{6} \\
                 &= \frac{(k^2 + k)(2k + 1)}{6} \\
                 &= \frac{2k^3 + k^2 + 2k^2 + k}{6} \\
                 &= \frac{2k^3 + 3k^2 + k}{6}
        \end{align*}

    \item[(c)] Express \( P(k + 1) \)
    
        \answer

        \begin{align*}
            P(k + 1) &= \sum_{j=1}^{k + 1} j^2 \\
                 &= \frac{(k + 1)(k + 1 + 1)[2(k + 1) + 1]}{6} \\
                 &= \frac{(k + 1)(k + 2)(2k + 3)}{6} \\
                 &= \frac{2k^3 + 9k^2 + 13k + 6}{6}         
        \end{align*}

    \item[(d)] In an inductive proof that for every positive integer \( n \), 
    \[
        \sum_{j=1}^n j^2 = \frac{n(n + 1)(2n + 1)}{6}
    \] 
    what must be proven in the base case?

        \answer

        We must prove \( n = 1 \), so 
        \begin{align*}
            \sum_{j=1}^1 j^2 &= 1^2 \\
                             &= 1
        \end{align*}

        \begin{align*}
            \frac{1(1 + 1)(2 \cdot 1 + 1)}{6} &= \frac{1(2)(3)}{6} \\
                                              &= 1
        \end{align*}

    \item[(e)] In an inductive proof that for every positive integer \( n \),
    \[
        \sum_{j=1}^n j^2 = \frac{n(n + 1)(2n + 1)}{6}
    \] 
    what must be proven in the inductive step?

        \answer

        For any positive integer \( k \), \( P(k) \implies P(k + 1) \). So for every \( k \geq 1 \), if 
        \[
            \sum_{j=1}^k j^2 = \frac{k(k + 1)(2k + 1)}{6}
        \],
        then 
        \[
            \sum_{j=1}^{k + 1} j^2 = \frac{(k + 1)(k + 1 + 1)[2(k + 1) + 1]}{6}
        \]

    \item[(f)] What would be the inductive hypothesis in the inductive step from your previous answer?
    
        \answer

        The inductive hypothesis would be
        \[
            \sum_{j=1}^k j^2 = \frac{k(k + 1)(2k + 1)}{6}
        \]

    \item[(g)] Prove by induction that for any positive integer \( n \), 
    \[
        \sum_{j=1}^n j^2 = \frac{n(n + 1)(2n + 1)}{6}
    \]

        \answer

        In part (d) we proved the base case. In part (e) we set up the inductive step. So,

        \begin{align*}
            \sum_{j=1}^{k + 1} j^2 &= \sum_{j=1}^{k} + (k + 1)^2 \\
                                   &= \frac{k(k + 1)(2k + 1)}{6} + (k + 1)^2, \text{ by the induction hypothesis} \\
                                   &= \frac{2k^3 + 3k^2 + k}{6} + (k + 1)^2, \text{ by part (b)} \\
                                   &= \frac{2k^3 + 3k^2 + k}{6} + \frac{6}{6}(k^2 + 2k + 1) \\
                                   &= \frac{2k^3 + 3k^2 + k + 6k^2 + 12k + 6}{6} \\
                                   &= \frac{2k^3 + 9k^2 + 13k + 6}{6}
        \end{align*}

        And we know that \( P(k + 1) = \frac{2k^3 + 9k^2 + 13k + 6}{6} \) from part (c).

\end{enumerate}

\subsection*{Section B: zyBooks Exercise 7.4.3; c}

\begin{enumerate}
    
    \item[(c)] Prove that for n \( geq \) 1, 
    \[
        \sum_{j=1}^n \frac{1}{j^2} \leq 2 - \frac{1}{n}
    \]

        \answer

        Step 1: The base case is \( n = 1 \). 
        
        \begin{align*}
            \sum_{j=1}^1 \frac{1}{1^2} &= \frac{1}{1} \\
                                       &= 1
        \end{align*}

        \begin{align*}
            2 - \frac{1}{1} &= 1
        \end{align*}

        \( 1 \leq 1 \), so the base case is satisfied.

        \medskip

        Step 2: For \( k \geq 1 \), we want to show that 
        \[
            \sum_{j=1}^{k+1} \frac{1}{j^2} \leq 2 - \frac{1}{k + 1}
        \]

        \begin{align*}
            \sum_{j=1}^{k+1} \frac{1}{j^2} &= \frac{1}{(k + 1)^2} + \sum_{j=1}^{k} \frac{1}{j^2} \\
                                           &= \frac{1}{(k + 1)^2} + \left( 2 - \frac{1}{k} \right)
        \end{align*}
        
        So
        \begin{align*}
            \sum_{j=1}^{k+1} \frac{1}{j^2} &\leq \frac{1}{(k + 1)^2} + \left( 2 - \frac{1}{k} \right) \\
                &\leq \frac{1}{k(k + 1)} + \left( 2 - \frac{1}{k} \right) \\
                &\leq \frac{1}{k} \frac{1}{k + 1} - \frac{1}{k} + 2 \\
                &\leq \frac{1}{k} \left( \frac{1}{k + 1} - 1 \right) + 2 \\
                &\leq 2 - \frac{1}{k} \left( 1 - \frac{1}{k + 1} \right) \\
                &\leq 2 - \frac{1}{k} \left( \frac{k + 1}{k + 1} - \frac{1}{k + 1} \right) \\
                &\leq 2 - \frac{1}{k} \left( \frac{k + 1 - 1}{k + 1} \right) \\
                &\leq 2 - \frac{1}{k} \left( \frac{k}{k + 1} \right) \\
                &\leq 2 - \left( \frac{1}{k + 1} \right) \\
        \end{align*}

        We know that \( \frac{1}{(k + 1)^2} \leq \frac{1}{k(k + 1)} \) since \( k \geq 1 \). So we can make that replacement in line 2.
\end{enumerate}

\subsection*{Section C: zyBooks Exercise 7.5.1; a}

\begin{enumerate}
    
    \item[(a)] Prove that for any positive integer \( n \), 4 evenly divides \( 3^2n - 1 \).
    
        \answer

        Step 1: The base case is \( n = 1 \). \( 3^{2(1)} - 1 = 8 \), which is divisible by four.

        \medskip

        Step 2: In the inductive step, we want to prove that for \( k \geq 1 \), 4 evenly divides \( 3^{2(k+1)} - 1 \). The induction hypothesis \( 3^2k - 1 = 4m \) is equivalent to \( (3^2)^k - 1 = 9^k - 1 = 4m \).

        \begin{align*}
            3^{2(k+1)} - 1 &= 9^{k+1} - 1 \\
                           &= 9 \cdot 9^k - 1 \\
                           &= 8 \cdot 9^k + 9^k - 1 \\
                           &= 8 \cdot 9^k + 4m, \text{ by the induction hypothesis} \\
                           &= 4(2 \cdot 9^k + m)
        \end{align*}

        Since \( k \geq 1 \) and \( m \) are integers, the inner expression is an integer as well.
\end{enumerate}

\end{document}